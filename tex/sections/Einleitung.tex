\section{Einleitung}

Ein grosses Thema der Hochfrequenztechnik sind Antennen. Generell können Antennen als Leiter angesehen werden, welche sich identisch zur Leitungstheorie verhalten. Jedoch gibt es viele verschiedene Ausführungen mit unterschiedlichsten Eigenschaften. Dieser Bericht ist als Weiterführung zu dem hf1-Bericht gedacht, bei welchem viele Antennen mit \textit{CST Studio Suite} simuliert wurden. Der hf2-Bericht spezialisiert sich auf Yagi-Antennen. Neben der theoretischen Arbeit soll selber eine Yagi-Antenne für \SI{144.125}{MHz} konstruiert und gebaut werden. Hierfür wurde zuerst die Theorie erarbeitet. Mit den berechneten Werten für die Yagi-Anntene werden Simulationen mit dem Programm \textit{CST Studio Suite} durchgeführt, welche sich vor allem auf das Strahlungsdiagramm der Antenne beziehen. Als Abschluss wird die Antenne im Labor ausgemessen und verifiziert.