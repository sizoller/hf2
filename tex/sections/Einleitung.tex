\section{Einleitung}

Ein grosses Thema der Hochfrequenztechnik sind Antennen. Generell können Antennen als Leiter angesehen werden, welche identisch zur Leitungstheorie gelten. Jedoch gibt es viele verschiedene Ausführungen welche alle sich unterschiedlich verhalten. Dieser Bericht ist als Weiterführung zu dem hf1-Bericht gedacht. Der hf2-Bericht spezialisiert sich auf Yagi-Antennen. Neben der theoretischen Arbeit soll selber eine Yagi-Antenne für 144MHz konstruiert und gebaut werden. Hierfür wurden zuerst die Theorie erarbeitet. Mit den berechneten Werten für die Yagi-Anntene wird eine Simulation mit dem Programm CST durchgeführt, welche sich vor allem auf die Direktivität der Antennen beziehen. Als Abschluss wird die Antenne noch ausgemessen.