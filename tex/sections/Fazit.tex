\section{Fazit}\label{sec:Fazit}

Als Fazit kann gezogen werden dass die Simulationen und der Bau der Yagi-Antenne ein Erfolg war. Es wurde eine Grenzfrequenz von \SI{144.125}{MHz} angestrebt, wobei die Simulation und Messung mit den \SI{133}{MHz} nur leicht davon abweichen. Während den Simulationen wurde viel über das Verhalten der Antenne gelernt, wobei es sehr viel mehr zu erlernen gibt. Vor allem für eine Optimierung der Antenne würden sich noch viele Möglichkeiten bieten, da sich die Werte aus der Theorie und der Praxis um einige Prozent abweichen. Zum Beispiel könnte der Dipol verkürzt werden und mit dieser sich neu ergebender Wellenlänge die Berechnungen nochmals durchgeführt werden. Somit kann ein genaueres Erreichen der \SI{144.125}{MHz} angestrebt werden. 

Ein weiteres Erfolgserlebnis war das Empfangen von Radiosendern mit der gebauten Antenne zusammen mit einem \textit{FUNcube Dongle}. Dabei konnte das Signal klar und mit besserem Gewinn empfangen werden. Das Rauschen war dabei minimal und es konnte angemessen Musik gehört werden.

Gesamthaft war der Lerngewinn sehr hoch und zusammen mit dem Bericht aus dem \textit{hf1}-Unterricht konnte sehr viel über Antennen und Simulationen mit dem \textit{CST Studio Suite} gelernt werden.
